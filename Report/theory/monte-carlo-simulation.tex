We make decisions every day and before during that, whether conscious or subconscious we are calculating risks of doing just that. It is impossible to predict what the future holds for us, but the Monte Carlo simulation will let you see which possible outcomes you can expect from the different decisions.
A Monte Carlo Simulation will let you know all possible outcomes which can occur and how possible it is that that outcome will be the one.
Basicly it is like saying what if thousands of times and seeing what happens and how often something happens.
If we were to roll a dice and find out what the possibility is that two dices rolls 8. We would throw the dices 100 times, and for each time we would record the throw. That way we have how many times we roll each of the outcomes and therefore we can predict the possibility of that outcome.
Lets say we rolled 8 12 times out of a 100 rolls. The possibility of rolling 8 is 18%. But instead of rolling the dice ourselve we could just simulate it by using the computer to simulate a 100 throws.
In poker this could maybe be used to predict what cards are popping up in the flop and so what are our chances of having a hand that is good.
Another alternative could be to determine whether a player is an experienced player or a beginner. If we were to look how often a player wins versus how many times you win, it could be a guideline to see how big a chance you have of winning the game.
