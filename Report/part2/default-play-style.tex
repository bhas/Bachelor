\section{Learning default play style}
\label{sec:part2}
As we cannot have a bot which can outplay the opponents from first round we will have to create a default player. The default player will have the task of lowering our costs while the system gathers information about the opponent so the system can adapt to his style of play. There are different ways of how to achieve this, firstly we could simply just hardcore a decent way of playing into the game and make decisions based on the cards only. The other method is to implement a neural network and train it till we are satisfied with the results. The neural network seemed like the best idea for further development so this will be our way to go.
\subsection{Design}
\subsection{Test}
\subsection{Discussion}
When our artificial intelligence starts out by playing the games of poker it doesn't have any kind of information about the opponents. So to make sure that the bot wont just go keep throwing away the bank roll. Instead we wanted the bot to minimizes our losses so that we would still have a decent bank roll when we have gathered information about the opponents so that the bot could make qualified guesses at what move would be the most appropriate in terms of the current opponent. The default player was never meant to be on the same level as a human player, but humans are only in a slightly better position than the default bot. The human player can like the bot only see the hole and community cards, but a human is also able to make a profile of the bot in their head. This means that they can learn how the bot decides and exploit this. 
This is one of the reasons that we chose to go with a neural network. In a neural network we are able to of course train the network to make correct decisions based on the targeted output that we give it.
But as the game proceeds the neural network has an input which will weight when we have enough information about the opponents to shift our gameplay from the default play style to a more adaptive one.
The time that it takes a human player to learn about the bot and adapt to it, should be the same for the bot. So if we imagine a human player who is good enough to decipher the way the bot is playing and adapt to it. When the human player does that, the bot should also have started to change its ways. Slowly as the bot learn it will adapt more and more to the opponent so when we have enough information the bot will shift completely and disregard the default play style.
\subsection{Conclusion}