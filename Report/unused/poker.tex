\section*{Texas hold'em poker}
Poker is arguably one of the most popular gambling games. It combines psychology, statistic, and luck into a single game with simple rules.  It is played with a standard deck of 52 playing cards. There are multiple variances of poker but this section only describes the basic rules of Texas hold'em limit poker. For a more in-depth description of the Texas hold`em limit rules see \cite{poker-rules}.

\subsubsection*{Player actions}
\begin{tabular}{ p{0.1\linewidth} p{0.9\linewidth} }
  \textbf{Fold} & The player gives up and the player is out until the start of the next round. This action is always possible.\\
  \textbf{Check} & The player does not bet any money. This action is only possible if no other player has placed a bet.\\
  \textbf{Bet} & The player bets an amount equal to the blind. This action is only possible if no other player has placed a bet.\\
  \textbf{Call} & The player place a bet equals to the highest bet. This action is only possible if another player has placed a bet.\\
  \textbf{Raise} & The player increases the highest bet by a big blind thus having the new highest bet. This action is only possible if another player has placed a bet.\\
  \textbf{All-in} & The player bets all his money. The player is still in the game and plays for the pot but he will not be able to act any more. This action is only possible if you cannot afford to call, bet or raise.\\
\end{tabular}

\subsubsection*{Keywords}

\begin{tabular}{ p{0.18\linewidth} p{0.82\linewidth} }
\textbf{Dealer} & Players take turn being the dealer. The dealer changes in the beginning of each round. The dealer is used to determine the order of which the players perform their action. The order is clockwise starting with the player to the left of the dealer. \\
\textbf{Blind} & A fixed amount that the first two players are required to pay. The first player pays small blind (half blind) and the second pays big blind (full blind).\\
\textbf{Hole cards} & The pair private cards that is dealt to each player. \\
\textbf{Community cards} & The five public cards that are dealt to the table and are shared between all players.\\
\textbf{Bidding round} & The phase where the players take turn performing an action. A bidding round will occur in every game state.\\
\textbf{Action} & An action can be performed when it is the players turn to act.\\
\textbf{Pot} & The sum of all the bids that have been placed in the round. The winner of the round takes the pot.\\
\textbf{Chips} & The amount of money a player has.\\
\textbf{Round} & A round consists of 4 states: pre-flop, flop, turn, and river. In each round new cards are dealt.\\
\textbf{Pre-flop} & The first game state. In this state the blinds are paid and the hole cards are dealt.\\
\textbf{Flop} & The second game state. In this state the first three community cards are dealt.\\
\textbf{Turn} & The third game state. In this state the fourth community card is dealt.\\
\textbf{River} & The final game state. In this state the fifth community card is dealt. After the bidding round a showdown will take place.\\
\textbf{Showdown} & The phase where all players who have not folded show their whole cards and the winner is determined.\\
\textbf{Hand} & The best combination of five cards a player can make from his hole cards and the community cards. See table \ref{tab:poker-ranks}.
\end{tabular}

\subsubsection*{Game play}
A Texas hold`em poker game consists of multiple rounds and each round is divided into four game states: pre-flop, flop, turn and river. In each game state a bidding round will take place after the cards are dealt. 

In the bidding round the players will perform their actions.

If the players make it to the showdown they reveal their hole cards. The player with the strongest hand will then win the pot. In case of a draw the pot is split. Then the rounds is finished and a new round starts.
 
The amount of chips is the main indicator of how well the player is doing. The main goal of all players is to increase amount of their chips.

\subsubsection*{The bidding round}
The bidding round is where the players in turn perform their actions. The player to the left of the dealer is always the first to act. The turn then moves clockwise around the table. 

A player can play aggressive by betting or raising which will increase the cost for the other players to call. Likewise a player can also play defensive by calling or checking which won't increase the cost for others to call. If a player decides to fold he will lose what he betted that round. Whenever a player chooses to play aggressive all other players must either fold or call in order for the bidding round to stop. The bidding round continues until all players have called the aggressor or folded. If all other players fold and only one player is left he wins the pot.

\subsubsection*{Rules for determining the winner}
The rules for finding the best hand is quite simple. We refer to the rank of a card (2-A) as the card rank and the rank of the hand (see table \ref{tab:poker-ranks}) as the rank. First the rank of the  is found. The order of the cards is insignificant. Starting from the top (the best hand) in table \ref{tab:poker-ranks} we have the explanations:\\
\textbf{Royal Flush} is having a straight all in the same suit. The highest card also have to be an ace.\\
\textbf{Straight Flush} is the same as a royal flush except there is no requirement for the highest card.\\
\textbf{Four of a kind} is having four cards with the same card rank \\
\textbf{Full house} is having three of a kind and a pair\\
\textbf{Flush} is having five cards with the same suit\\
\textbf{Straight} is having five cards in a row (e.g 10-A or 4-9)\\
\textbf{Three of a kind} is having three cards with the same card rank \\
\textbf{Two pairs} is having two pairs\\
\textbf{One pairs} is having two cards with the same card rank. \\
\textbf{High card} is the highest card. \\


\begin{table}[H]
  \center
  \begin{tabular}{ | l | l | l | }
  	\hline
  	rank & name & example hand \\
  	\hline                       
    1 & Royal flush & A\clubsuit ~ K\clubsuit ~ Q\clubsuit ~ J\clubsuit ~ 10\clubsuit \\
    2 & Straight flush & 7\clubsuit ~ 6\clubsuit ~ 5\clubsuit ~ 4\clubsuit ~ 3\clubsuit \\
    3 & Four of a kind & K\clubsuit ~ K\spadesuit ~ K\diamondsuit ~ K\heartsuit ~ 10\clubsuit \\
    4 & Full house & K\clubsuit ~ K\spadesuit ~ K\diamondsuit ~ Q\heartsuit ~ Q\clubsuit \\
    5 & Flush & K\heartsuit ~ Q\heartsuit ~ 5\heartsuit ~ 3\heartsuit ~ 2\heartsuit \\
    6 & Straight & A\clubsuit ~ K\spadesuit ~ Q\diamondsuit ~ J\heartsuit ~ 10\clubsuit \\
    7 & Three of a kind & A\clubsuit ~ A\diamondsuit ~ A\spadesuit ~ J\clubsuit ~ 10\clubsuit \\
    8 & Two pairs & A\clubsuit ~ A\diamondsuit ~ 5\spadesuit ~ 5\clubsuit ~ 4\clubsuit \\
    9 & One pair & A\clubsuit ~ A\heartsuit ~ J\clubsuit ~ 9\spadesuit ~ 2\heartsuit \\
    10 & High card & A\clubsuit ~ K\diamondsuit ~ 6\heartsuit ~ 5\heartsuit ~ 3\heartsuit~ \\
  	\hline   	
  \end{tabular}
  \caption{Ranks of different hands in Texas hold'em poker sorted best to worst. \label{tab:poker-ranks}}
\end{table}
