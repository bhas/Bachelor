\section*{Introduction}
Poker is the most popular card game in the world \cite{poker-popular}. It involves intelligence, psychology and luck. Poker has a lot of hidden informations which requires the players to make decisions based on qualified guesses.  

In a game of poker analysing your opponents is a major part. Being able to predict your opponents hands, future actions and reactions can give you a huge advantage. Top players are able to read their opponents and adapt their strategy in order to gain an advantage. Most rounds are won before the showdown simply by one player outplaying the others. Another big element of poker is statistics. Since you don't know the cards that will be dealt throughout the game players must calculate the likelihood of their hand ending up being the winning hand.  

For humans it is often easy to recognise various patterns whereas it can be very challenging for a computer. On the other hand computers can perform thousands of computations in a matter of seconds. Even for most humans players it is really hard to read your opponents and it is even harder for a computer.\\

The goal of this thesis is to see if it is possible to train a computer to play poker and adapt to its opponents strategy. In order to achieve this goal we first need to answer the following questions:

\begin{enumerate}
  \item How can we predict the probability of ending up with the winning hand?
  \item Can the computer learn to play poker by observing humans playing poker, and if so how can we implement it?
  \item How can the computer adapt to the opponents strategies?
\end{enumerate}