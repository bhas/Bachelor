\section{Learning to adapt to opponents}
\label{sec:part3}

In the previous chapter we created an algorithm to develop the default strategy for the APC. The default strategy is one of the two strategies that will be used to determine the actions of the APC. The other strategy is the adaptive strategy. \\

The goal of this chapter is to learn the APC to create an algorithm to develop the adaptive strategy by solving problem statement 3.

\vspace{4mm}
\begin{statementBox2}{Problem statement 3}
  How can one develop a strategy adapted to the opponents?
\end{statementBox2}
\vspace{4mm}

In poker there is no such thing as a single optimal strategy. Every strategy has weaknesses and therefore the optimal strategy is one that takes advantage of the weaknesses of the strategies of the opponents. In order to take advantage of the opponents strategies one must first understand their strategy. In poker understanding the opponents is one of the most important key elements of the game. 

Once one understand the strategy of the opponent one has to adapt oneself's strategy.



\subsubsection{Player modeling}
Player modeling is a loosely defined concept and may vary from one context to another. The concept of player modeling is to make a computational model of a player. This model includes game related attributes, such as play style and preferences, as well as non-game related attributes, such as cultural background, gender, and personality. All decisions of the player are ultimately made on the basis of these attributes. 

Player modeling is used to describe or predict the players decisions, reasoning and reactions. In the field of artificial intelligences the human player is the most used model for developing computer players. Understanding the reason behind every choice of a player will not only bring a better understanding of the player but also a better understanding of the game and its mechanics.

Since the player model can easily become extremely extensive one normally only includes the relevant attributes of the player.

If player can optimally model an opponent, that player will have a much greater chance of being successful. A player who have the same amount of knowledge regarding the fundamentals of poker as the opponent, but lacks the ability to change strategy throughout the game will probably not win as much as the opponent.

\subsection{How can we model a player?}
Quite a few decisions needs to be made when beginning to player model. It is crucial to figure out which attributes are relevant to create the specific type of player model that one wishes to implement.
The attributes we find relevant is the following:
\begin{enumerate}
  \item The opponents aggressiveness
  \item The opponents current aggressiveness
  \item The opponents bluff percentage
  \item The opponents reaction time
\end{enumerate}

Each of these attributes and the reasoning behind the choices is described below.\\

As being stated earlier, poker is a game of imperfect information. The betting strategy of a player is the only piece of information that an opponent is able to see.
But just because a player is choosing to raise, does not necessarily mean that the player has a hand that has a good chance of winning.
A player who knows the fundamentals of poker are aware that the way of betting could reflect what cards he is holding. A player will therefore often try to bet the opposite of what is logic in order to disguise his cards further.
It goes without saying that predicting the cards of an opponent is rather impossible. This is why we chose aggressiveness as one of the attributes for the player modeling.
The opponents aggressiveness can be calculated by looking at the history of the game. If an opponent raises in every game the system will address him as an aggressive player and calculate a value describing how aggressive he is. Likewise if the opponent is limiting his actions to bet/check then the aggressiveness value will decrease. 

The opponents current aggressiveness has been chosen in order to make APC prepared for any changes in the opponents characteristics. If an opponent has been playing safe the whole game and suddenly begins to raise, it is quite possible that he is either changing strategy or maybe he has received some good hole cards. If the opponent has a habit of raising each time he receives some good hole cards, it says a lot about his current strategy. However, the main focus will still be in regards to the player changing his strategy which APC will have to adapt to.

The opponents bluff percentage is chosen as a relevant factor in regards to a modeling of the opponent as it tells whether or not the opponent is playing solely from his point of view. If an opponent is playing solely from his own point of view and not thinking about the see-through he creates when doing this, then APC will have a great advantage. After several games, APC will notice a pattern for when the opponent is raising compared to the cards shown at the showdown. APC will then for the further games against this opponent be able to distinguish between the opponent bluffing or not.


The opponents reaction time may or may not be relevant when modeling the opponent. The reaction time used by the opponent when making a decision does not necessarily mean that the opponent is uncertain if he should do this or that. Many factors come into play when an opponent is having his turn. A delay in reaction can be due to the opponent being away from the keyboard (AFK), whether or not he should fold, bet or raise based on his hole cards, the community cards, APCs action among other things. 
\\

\subsection{How can we model a player dynamically?}
In order to make the player modeling dynamic we will have to introduce an ANN. The ANN can learn dynamically based on the inputs given. The network keeps getting new inputs that can be used in the current dataset for an opponent and should be added so that the network can learn each time the opponent makes a decision that gives us more data.

\subsection{Test}


\subsection{Discussion}

\subsection{Conclusion}