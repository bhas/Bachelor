\section{Learning to adapt to opponents}
\label{sec:part3}

In the previous section we created an algorithm for learning the default strategy for the computer. The default strategy is one out the two strategies that will be used to determine the actions of the computer. The other strategy is the adaptive strategy. \\

The goal of this section is to learn the computer to create an algorithm to develop the adaptive strategy by solving problem statement 3.

\vspace{4mm}
\begin{statementBox2}{Problem statement 3}
  How can the computer adapt to the opponents strategies?
\end{statementBox2}
\vspace{4mm}

In poker there is no such thing as a single optimal strategy. Every strategy has weaknesses and therefore the optimal strategy is one that take advantage of the weaknesses of the strategies of the opponents. In order to take advantage of the the opponents strategies one must first understand their strategy. In poker understanding the opponents is one of the most important key elements of the game.

\subsubsection{What is player-modeling?}
Player modeling is a loosely defined concept and may vary from one context to another. The concept of player modeling is to make a computational model of a player. This model includes game related attributes, such as play style and preferences, as well as non-game related attributes, such as cultural background, gender, and personality. All decisions of the player are ultimately made on the basis of these attributes. 

Player modeling is used to describe or predict the players decisions, reasoning and reactions. In the field of artificial intelligences the human player is the most used model for developing computer players. Understanding the reason behind every choice of a player will not only bring a better understanding of the player but also a better understanding of the game and its mechanics.

Since the player model can easily become extremely extensive one normally only includes the relevant attributes of the player.

\subsection{How can we model a player dynamically?}
To player model each player the system has to look at the games previous history while dynamic learning as the game proceeds.
This can be implemented using a neural network and will help us reach our goal of predicting the opponents cards, next move or as a minimum what strength their dealt hand has. By using neural network we can model a player throughout the game. First the neural network of course take many inputs but one of them would be about a specific player and their history, the choices they made, their chips, cards etc. From that the neural network will continue to receive inputs about the player so that player modeling can be dynamic.
Another big advantages of a neural network is that we are able to give a lot of inputs. These inputs will then be weighted by the system in order to come closer to our targeted output.
This will help us cut out all of the noise that occur and leave us with data that is relevant.

\subsection{Test}


\subsection{Discussion}

\subsection{Conclusion}