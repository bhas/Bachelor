\section{Learning to adapt to opponents}
\label{sec:part3}

In the previous chapter we created an algorithm to develop the default strategy for the APC. The default strategy is one of the two strategies that will be used to determine the actions of the APC. The other strategy is the adaptive strategy. \\

The goal of this chapter is to learn the APC to create an algorithm to develop the adaptive strategy by solving problem statement 3.

\vspace{4mm}
\begin{statementBox2}{Problem statement 3}
  How can one develop a strategy adapted to the opponents?
\end{statementBox2}
\vspace{4mm} 

In poker there is no such thing as a single optimal strategy. Every strategy has weaknesses and therefore the optimal strategy is one that takes advantage of the weaknesses of the strategies of the opponents. In order to take advantage of the opponents strategies one must first understand their strategy. In poker understanding the opponents is one of the most important key elements of the game. 

Once one understand the strategy of the opponent one has to adapt oneself's strategy.
The problem statement rises a few problematics when trying to solve it. For instance the first step is figuring out exactly which possibilities that are available which will suit the needs for the completion of the solution to the problem statement. The obvious solution is to implement player modeling which leads us on to the next problem.
When implementing player modeling, there are many different aspects that could be relevant. The challenge here is to distinguish  between what is relevant and what is not.



\subsubsection{Player modeling}
Player modeling is a loosely defined concept and may vary from one context to another. The concept of player modeling is to make a computational model of a player. This model includes game related attributes, such as play style and preferences, as well as non-game related attributes, such as cultural background, gender, and personality. All decisions of the player are ultimately made on the basis of these attributes. 

Player modeling is used to describe or predict the players decisions, reasoning and reactions. In the field of artificial intelligences the human player is the most used model for developing computer players. Understanding the reason behind every choice of a player will not only bring a better understanding of the player but also a better understanding of the game and its mechanics.

Since the player model can easily become extremely extensive one normally only includes the relevant attributes of the player.

If player can optimally model an opponent, that player will have a much greater chance of being successful. A player who have the same amount of knowledge regarding the fundamentals of poker as the opponent, but lacks the ability to change strategy throughout the game will probably not win as much as the opponent.

\subsection{How can we model a player?}
Quite a few decisions needs to be made when beginning to player model. It is crucial to figure out which attributes are relevant to the specific player model. When trying to model a player there is almost no limit to what could be included. A player model could include but is not limited to; the players childhood, if the player is hungry, distractions, the players body language etc. All of these and many more are what makes the player him. But as many of these are out of the scope of this project, this chapter will focus on a rather few but relevant attributes. 

The attributes that is relevant when modeling a poker player:
\begin{enumerate}
  \item The players aggressiveness
  \item The players bluff percentage
  \item The players decision delay
\end{enumerate}

Each of these attributes and the reasoning behind the choices is described below.\\

As being stated earlier, poker is a game of imperfect information. The betting strategy of a player is the only piece of information that an opponent is able to see.
But just because a player is choosing to raise, does not necessarily mean that the player has a hand that has a good chance of winning.
A player who knows the fundamentals of poker are aware that the way of betting could reflect what cards he is holding. A player will therefore often try to bet the opposite of what is logic in order to disguise his cards further.
It goes without saying that predicting the cards of an opponent is rather impossible. \\

The players aggressiveness can be split in two separate attributes; the overall aggressiveness and the current aggressiveness. We chose aggressiveness as one of the attributes for the player modeling as it could be a remedy in order to determine whether or not APC should fold or play on.
The opponents aggressiveness can be calculated by looking at the history of the game. If a player raises in every game the system will address him as an aggressive player and calculate a value describing how aggressive he is. Likewise if the player is limiting his actions to bet/check then the aggressiveness value will decrease. 

The players current aggressiveness can make APC prepared for any changes in the players characteristics. If a player has been playing safe the whole game and suddenly begins to raise, it is quite possible that he is either changing strategy or maybe he has received some good hole cards. If the player has a habit of raising each time he receives some good hole cards, it says a lot about his current strategy. However, the main focus will still be in regards to the player changing his strategy which APC will have to adapt to.

The players bluff percentage is chosen as a relevant factor in regards to modeling of the player as it tells whether or not the player is playing solely from his point of view. If a player is playing solely from his own point of view and not thinking about the see-through he creates when doing this, then APC will have a great advantage. After several games, APC will notice a pattern for when the player is raising compared to the cards shown at the showdown. APC will then for the further games against this player be able to distinguish between if the player is bluffing or not.


The delay in the players decision making may or may not be relevant when modeling the player. The time used by the player when making a decision does not necessarily mean that the player is uncertain if he should do this or that. Many factors come into play when a player is having his turn. A delay can be due to the player being away from the keyboard (AFK), whether or not he should fold, bet or raise based on his hole cards, the community cards, APCs action among other things. 
\\

\subsection{How can we model a player dynamically?}
In order to make the player modeling dynamic we will have to introduce a new ANN. The ANN can learn dynamically based on the inputs given. The network keeps getting new inputs that can be used in the current dataset for an opponent and should be added so that the network can learn each time the opponent makes a decision that gives us more data.

The adaptive part of APC will be using a multi-layer perceptron wtih the structure x-x-2. The supervised learning method will be used to make the network learn about the player by having an expected output. The initial part will be using the already developed default strategy from chapter 2 until the adaptive part has enough information about the player to choose a qualified action against the opponent.
It is possible to retrain the whole network each time a round is finished. That way the ANN can train with the new data that is available but this would be time very time consuming compared to the small amount of new data at each round.
A small amount of data may be insignificant to the training of the ANN, and therefore this is perhaps not the best approach.

The ANN will have the following inputs:

\begin{enumerate}
  \item The probability of the hole cards
  \item Chips in the pot
  \item Cost to call
  \item Number of opponents
  \item Opponents chips
  \item Opponents aggressiveness
  \item Opponents recent aggressiveness
\end{enumerate}


Each of these attributes and the reasoning behind the choices is described below.\\

The probability of the hole cards is always a part of the decision making when a player is choosing whether or not to play the hand or not. The problem with this input is that a bad hand can have a much greater value later in the game. Likewise a good hand in the beginning can become very bad when the community cards are revealed. Therefore this is not the only input that is relevant when making a decision.


The chips in the pot and the cost to call are two inputs that is very valuable to the player when making a decision. When a player is deciding he will look at the pot to see how much is able to win compared to what he could loose. This is a major factor in decision making. Of course a player wants to keep the risks of losing to a minimal but the chance of winning to a maximum.

The number of opponents is also a relevant input as the choice a player takes depends loosely on how many opponents he is currently betting against. If a player is playing against nine opponents, his chances of winning is less than if he were playing against only one opponent. But the amount of winnings is quite larger in the game with nine opponents as each of these opponents has to lay the blinds, and likewise they are able to bet and raise the pot.
Therefore a player with good cards have a few options, he could either play a little defensive hoping the others will try to bluff and thereby betting. This would allow our player to remain with disguised cards. Another option is to bet and revealing that he has a good hand, but that could lead to the other players to fold and he would loose profits that could have been. As the number of opponents can decrease throughout a round this is worth taking into account, as it will affect a players decisions.

The amount of chips each opponent have is important to know because it will affect the players decision. Since a player wants to win the biggest amount possible, the player will need to predict how the opponents will react to the bet. It goes without saying that an opponent with a low amount of chips is unlikely to call a raise that will have him go all-in and therefore the player will...



\subsection{Design}



\subsection{Test}


\subsection{Discussion}

\subsection{Conclusion}