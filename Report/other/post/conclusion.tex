\section{Conclusion}
In this thesis we have tried to implement an APC that can play Texas hold'em limit poker against up to nine opponents. The APC have been programmed in Java.

We started in chapter \ref{sec:part1} by answering problem statement 1:
\vspace{4mm}
\begin{statementBox2}{Problem statement 1}
How can APC determine the strength of a poker hand in any game state?
\end{statementBox2}
\vspace{4mm}

We define the strength of a poker hand as the probability of winning with that hand.
We have created a subsystem called the calculator which uses the Monte Carlo method. The calculator can, for a given hand and number of opponents, find an estimate of the true probability. 

It has an error of less than one percent and can calculate the result in $\sim$0,15 seconds.\\

Next, in chapter \ref{sec:part2}, we answered problem statement 2:
\vspace{4mm}
\begin{statementBox2}{Problem statement 2}
How can we develop a default strategy for APC without having information about the playing style of the opponents?
\end{statementBox2}
\vspace{4mm}

We use an ANN to observe real-life poker players in order for it to learn the strategy of the players. 

This was successful in the sense that we managed to make the ANN smarter through learning. But it did not fulfil our requirement of having a TNE of five percent or less after training. The closest we came was $\sim$14 \%, when the ANN only observed a single player. 
We further tested how the ANN would perform against another poker bot. 

When tested against another simple poker bot it lost roughly half its chips over 80 rounds. Due to imperfect data, we were not able to learn the ANN when to fold, which we believe is the main reason of its bad performance.\\

Finally we answered problem statement 3.
\vspace{4mm}
\begin{statementBox2}{Problem statement 3}
How can we further develop the APC's strategy to be able to adapt to the playing style of the opponent?
\end{statementBox2}
\vspace{4mm}
