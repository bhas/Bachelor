\section*{Introduction}
Poker is arguably the most popular card game in the world. It involves intelligence, statistics, psychology, and luck. Even though the game itself is fairly simple, it takes years of practice to master all elements of the game.

In poker, reading an opponent refers to the process of figuring out the opponents strategy based on their actions. The top players are able to read their opponents and adapt to their strategy in order to gain an advantage. Players often mix in unorthodox actions or change strategies in order to mislead the opponents.

Another big element of poker is statistics. Since the players do not know the cards that will be dealt throughout the game, players must calculate the likelihood of winning. \\

The goal of this thesis is to develop an adaptive poker computer (APC), that is capable of adapting to the strategies of the opponents. The APC will be programmed to play a variation of poker called Texas hold'em limit poker. For the scope of this thesis the APC will only play against a single opponent, also called heads-up in poker.

For the readers unfamiliar with Texas hold'em limit poker the following section describes the basic rules and flow of the game.

\subsection*{Texas hold'em poker}
This section only describes the basics of Texas hold'em limit poker. For a more in-depth description see \cite{poker-rules}. 

Poker is played with a standard deck of 52 cards. This thesis only concerns Texas Hold'em limit poker. \cite{poker-rules}.

\subsubsection*{Game play}
A Texas hold`em poker game consists of multiple rounds. In each round each player is dealt two private cards called hole cards that are hidden for the opponents. Five public cards, called community cards, that is visible for everybody are dealt as the round progresses. 

A round is divided into four game states: pre-flop, flop, turn, and river. Each game state starts with cards being dealt and then the bidding begins. In the pre-flop the dealer will deal the hole cards to each player. In the flop the first three out of five community cards are dealt and in the turn and river the fourth and fifth community card are dealt respectively. 

If only one player is left after the bidding that player wins the pot, otherwise the game continues to the next game state. If multiple players are still left after the bidding succeeding the river, a showdown will start. During showdown each player still in the game will reveal their hole cards and a winner will be found. The winner wins the pot. In case of a draw the pot is split between the winners.

The amount of chips is the main indicator of how well the player is doing. The main goal for the players is to increase their amount of chips.

\subsubsection*{The bidding}
The bidding is where the players in turn perform their actions. The actions include: call, bet, raise, fold, check, all-in. The player to the left of the dealer is always the first to act.

A player can play aggressively by betting or raising which will increase the cost for the other players. Likewise a player can also play defensively by calling or checking which will not increase the cost for other players. If a player decides to fold he will lose what he betted that round. Whenever a player chooses to play aggressively all other players must either fold or call in order for the bidding round to stop. The bidding continues until all players have called the aggressor or folded.

\subsubsection*{Rules for determining the winner}
The rules for finding the best hand a quite simple. Each player has to create the best possible hand choosing five of the seven card available cards (hole cards and the community cards). Each playing card has a card rank (2, ... , Q, K, A) and a suit (diamond, heart, club, or spade). The possible ranks of a hand can be seen in table \ref{tab:poker-ranks}. 

First the rank of the hand of every player is found. In case two players has the same rank, the winner will be determined by more advanced rules. These rules is not described in this section as they are irrelevant for this thesis. An case the hands are still even after using the advanced rules the round results in a draw and the pot is split between the players.

Starting from the bottom (the worst hand) of table \ref{tab:poker-ranks} we have the explanations:\\
\textbf{High card} is the highest card.  \\
\textbf{One pairs} is having two cards with the same card rank. \\
\textbf{Two pairs} is having two pairs\\
\textbf{Three of a kind} is having three cards with the same card rank \\
\textbf{Straight} is having five cards in a row (e.g 10-A or 4-9)\\
\textbf{Flush} is having five cards with the same suit\\
\textbf{Full house} is having three of a kind and a pair\\
\textbf{Four of a kind} is having four cards with the same card rank \\
\textbf{Straight Flush} is the same as a royal flush except there is no requirement for the highest card.\\
\textbf{Royal Flush} is having a straight all in the same suit. Furthermore the highest card have to be an ace.\\

\begin{table}[H]
  \center
  \begin{tabular}{ | l | l | l | }
  	\hline
  	rank & name & example hand \\
  	\hline                       
    1 & Royal flush & A\clubsuit ~ K\clubsuit ~ Q\clubsuit ~ J\clubsuit ~ 10\clubsuit \\
    2 & Straight flush & 7\clubsuit ~ 6\clubsuit ~ 5\clubsuit ~ 4\clubsuit ~ 3\clubsuit \\
    3 & Four of a kind & K\clubsuit ~ K\spadesuit ~ K\diamondsuit ~ K\heartsuit ~ 10\clubsuit \\
    4 & Full house & K\clubsuit ~ K\spadesuit ~ K\diamondsuit ~ Q\heartsuit ~ Q\clubsuit \\
    5 & Flush & K\heartsuit ~ Q\heartsuit ~ 5\heartsuit ~ 3\heartsuit ~ 2\heartsuit \\
    6 & Straight & A\clubsuit ~ K\spadesuit ~ Q\diamondsuit ~ J\heartsuit ~ 10\clubsuit \\
    7 & Three of a kind & A\clubsuit ~ A\diamondsuit ~ A\spadesuit ~ J\clubsuit ~ 10\clubsuit \\
    8 & Two pairs & A\clubsuit ~ A\diamondsuit ~ 5\spadesuit ~ 5\clubsuit ~ 4\clubsuit \\
    9 & One pair & A\clubsuit ~ A\heartsuit ~ J\clubsuit ~ 9\spadesuit ~ 2\heartsuit \\
    10 & High card & A\clubsuit ~ K\diamondsuit ~ 6\heartsuit ~ 5\heartsuit ~ 3\heartsuit~ \\
  	\hline   	
  \end{tabular}
  \caption{Ranks of different hands in Texas hold'em poker sorted best to worst. \label{tab:poker-ranks}}
\end{table}


\subsection*{Artificial intelligence and poker}
In the field of artificial intelligence, games are interesting because of their well-defined game rules and success criteria.
Computers have already mastered some of the popular games, one example is the chess computer Deep Blue which won against Garry Kasparov, the world champion of chess at the time.
Chess is a game of perfect information, as no information is hidden from the players.

Since then, the interest of the artificial intelligence research has shifted towards games with imperfect information. These types of games presents new challenges such as deception and hidden information. Poker is an example of a game with imperfect information.\\

In may 2015 the contest \textit{Brains Vs. Artificial Intelligence} \cite{brain-vs-ai} was held with four of the best poker players in the world. Each of the players played $\sim$20.000 hands of Texas Hold'em no-limit heads-up against Claudico, the world's best poker computer at the time. Claudico was able to beat one of the four players, which proves that artificial intelligence in regards to poker have come a long way.\\

Developing an algorithm capable of playing poker is not only limited to the domain of poker, but can end up having a future applications in other domains as well.

\begin{quotation}
\textit{``Bowling says the findings in this new research are especially valuable because they give us a hint at the scale of problems AI can solve. \ldots ~Solving a game as complex as heads-up limit Texas hold ’em could mean a breakthrough in our conception of how big is too big"} \cite{quote}
\end{quotation}

In essence a game presents a challenge for the player to solve. How the player approaches this challenge and what strategy the player uses to solve it, is what determines the players success.\\

In order to achieve the goal of developing an APC, we will try to answer the following problem statements:

\vspace{4mm}
\begin{statementBox2}{Problem statements}
\begin{enumerate}
    \item \label{itm:q1} How can APC determine the strength of a poker hand at any game state? \label{itm:ps1}
    \item \label{itm:q2} How can we develop a default strategy for APC without having information about the opponents? \label{itm:ps2}
    \item \label{itm:q3} How can we further develop the APC's strategy to be able to adapt to the playing style of the opponent? \label{itm:ps3}
  \end{enumerate}
\end{statementBox2}
\vspace{4mm}

Our thesis is divided into three chapters each of which focuses on one of the three problem statements.

In chapter \ref{sec:part1} we find a solution to problem statement one. We develop a subsystem which is able to estimate the probability of winning for any set of hole cards in any poker state. The subsystem can calculate the probability of winning with an error percentage of one percent and it takes less than a second on average.

In chapter \ref{sec:part2} we answer problem statement two. We implement a self-learning algorithm using artificial neural networks and use it to observe data from real-life poker games in order to learn the strategies of the players. The self-learning algorithm did learn the strategy of the players to some degree. However it were not able to compete on the same level as other bots.

In chapter \ref{sec:part3} ...
