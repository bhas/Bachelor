\section*{Introduction}
Poker is the most popular card game in the world \cite{poker-popular}. It involves intelligence, psychology, and luck. Poker has a lot of hidden information which requires the players to make decisions based on qualified guesses.  

Reading a player refers to the process of figuring out the players strategy and is a major element of poker. The top players are able to read their opponents and adapt to their strategy in order to gain an advantage. Most rounds are won before the showdown simply by one player outplaying the others. 

Another big element of poker is statistics. Since the players do not know the cards that will be dealt throughout the game, players must calculate the likelihood of their hand ending up being the winning hand. The more likely the player is to win the more aggressive he can play to maximize his profit.\\


In the field of artificial intelligence research games are interesting because of their well-defined game rules and success criteria.
Computers have already mastered some of the popular games. One example is the chess bot Deep Blue which won against Garry Kasparov, the world champion of chess at that time.
Chess is a game of perfect information, as no information is hidden from the players.
Since then the interest of the artificial intelligence research has shifted towards games with imperfect information. These types of games presents new challenges such as deception and hidden information. 

Developing an algorithm capable of playing poker is not only limited to the domain of poker but can end up having a future applications in other domains as well. In essence a game presents a challenge for the players to solve. How the player approaches the challenge and what strategy uses to solve it, are what determines the players success. Likewise in any other domain a person will be presented with a number of challenges which each requires the person to develop a strategy.\\

In may 2015 a contest was held with four of the best professional poker players in the world. Each of the players had to play 20.000 hands of heads-up no-limit Texas Hold'em against the currently best poker bot in the world Claudico. Claudico were able to end up with more chips than one of the four players which proves that artificial intelligence in regards to poker have come a long way.


The goal of this thesis is to see if it is possible to train a computer to play poker and adapt to its opponents strategy. In order to achieve this goal we first need to solve the following problem statements:


\vspace{4mm}
\begin{statementBox2}{Problem statements}
\begin{enumerate}
    \item \label{itm:q1} How can we predict the probability of ending up with the winning hand? 
    \item \label{itm:q2} Can the computer learn to play poker by observing humans playing poker, and if so how can we implement it? 
    \item \label{itm:q3} How can the computer adapt to the opponents strategies?
  \end{enumerate}
\end{statementBox2}
\vspace{4mm}

Our thesis is divided into three sections each of which focuses on one problem statement.

In section~\ref{sec:part1} we develop a subsystem which is able to estimate the probability of winning for any set of hole cards in any poker state. The subsystem can calculate the probability of winning with an error percentage of one percent and it takes less than a second on average.

In section~\ref{sec:part2} ...

In section~\ref{sec:part3} ...
